Основная идея проекта -\/ не ограничивать игрока в создании заклинаний. Игрок может создавать заклинания, опираясь на физические законы окружающего мира.

\subsection*{Основные концепции проекта\-:}


\begin{DoxyItemize}
\item Проект -\/ игра в жанре R\-P\-G
\item Двумерный игровой мир в представлении пользователя
\item Трёхмерный игровой мир с фиксированной глубиной со стороны логики игры
\item Основной “фишкой” игры является возможность построения -\/ произвольных заклинаний
\item Заклинания строятся в отдельном трёхмерном интерфейсе, управление глубиной на колёсико мыши или wasd+shift+space
\item Заклинания представляют из себя трёхмерные графы, состоящие из узлов и связывающих их рёбер
\item Заклинания завязаны на законах физики и связаны с температурой объектов и кинетикой
\item Создание заклинаний базируется на схемотехнике
\item Приоритет в разработке отдаётся созданию движка создания заклинаний
\item Прототип схож с игрой Terraria
\item Разрушаемый мир
\item Деление мира на тайлы
\end{DoxyItemize}

\subsection*{Примеры заклинаний\-:}


\begin{DoxyItemize}
\item Огненный шар
\item Отталкивание
\end{DoxyItemize}

\subsection*{Документация\-:}

Автоматическая документация генерируется с помощью Doxygen\-: \href{https://bender-wardrobe.github.io/Recast/html/}{\tt https\-://bender-\/wardrobe.\-github.\-io/\-Recast/html/}

\subsection*{Команда \char`\"{}Шкаф Бендера\char`\"{}\-:}


\begin{DoxyItemize}
\item Олег Морозенков
\item Василий Дмитриев
\item Михаил Волынов
\item Юрий Голубев
\item Куликов Никита 
\end{DoxyItemize}